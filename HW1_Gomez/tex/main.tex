\documentclass[a4paper, 11pt]{exam}
\usepackage[T1]{fontenc}
\usepackage{titling}
\usepackage{url}
\usepackage{amsmath,amsthm,amssymb}
\usepackage{graphicx}
\usepackage{graphics}
\usepackage{listings}
\usepackage[dvipsnames]{xcolor}
\usepackage{tabularx}
\usepackage{ragged2e}
\usepackage{courier}
\usepackage{textcomp}
\usepackage{circuitikz}
\usepackage{tikz}
\usepackage{enumitem}
\usepackage{karnaugh-map}
\usepackage{bytefield}
\usepackage{mathrsfs}
\usepackage{cancel}
\usepackage[linesnumbered,ruled,vlined]{algorithm2e}
\usepackage{hyperref}

\newcommand{\invlaplace}[1]{%
\mathscr{L}^{-1}\left\{#1\right\}
}
\newcommand{\laplace}[1]{%
\mathscr{L}\left\{#1\right\}
}
\newcommand{\fourier}[1]{%
\mathscr{F}\left\{#1\right\}
}

\newcommand{\ztransform}[1]{%
\mathscr{Z}\left\{#1\right\}
}

\newcommand{\subtitle}[1]{%
  \posttitle{%
    \par\end{center}
    \begin{center}\large#1\end{center}
    }%
}

\usepackage{environ}

\NewEnviron{eqnsection}[2]{%
  \newcommand{\myvspace}{#1}%
  \vspace{\myvspace}%
  \begin{align*}
  \intertext{#2}
  \BODY
  \end{align*}%
  \vspace{\myvspace}%
}



\newlist{myenumerate}{enumerate}{2}
\setlist[myenumerate,1]{label=\roman*)}
\setlist[myenumerate,2]{label=\alph*)}



\newcommand\tab[1][1cm]{\hspace*{#1}}

\renewcommand{\labelenumi}{\alph{enumi})}

\title{Homework Assignment \#1}
\subtitle{ECE 6530: Digital Signal Processing \\
\today\\}
\author{ Miguel Gomez U1318856\\
\textbf{Homework set \#1}}
\date{Due Date: Sep 1, 2023\\
(100 points)}

\begin{document}
\maketitle
\noindent
\section{}
1.3 - Determine whether or not each of the following signals is periodic. In case a signal is periodic, specify its fundamental period.
\begin{enumerate}
    \item $x_a(t) = 3cos(5t + \frac{\pi}{6})$
    \item $x(n)   = 3cos(5n + \frac{\pi}{6})$
    \item $x(n)   = 2e^{j(\frac{n}{6} - \pi)}$
    \item $x(n)   = cos(\frac{n}{8})cos(\frac{n\pi}{8})$
    \item $x(n)   = cos(\pi \frac{n}{2}) - sin(\pi \frac{n}{8}) + 3cos(\pi \frac{n}{4} + \frac{\pi}{3})$
\end{enumerate}
\hrule 
\text{}\\
Determining periodicity requires us to examine the following:
\begin{equation}
    x_a(t) = x_a(t+T)
\end{equation}
\begin{equation}
    x(n) = x(n+N)
\end{equation}

\begin{eqnsection}{-3em}{a) Consider the following expression to determine if $x_a(t)$ is periodic:}
    \intertext{We can expand $x_a(t + T) $ as follows:}
    x_a(t + T) &= 3\cos\left(5(t + T) + \frac{\pi}{6}\right) \\
    &= 3\cos\left(5t +5T + \frac{\pi}{6}\right) \\
    \intertext{We know that $ 2\pi ft = 5t $, so $ f = \frac{5}{2\pi} $ and $ T = \frac{2\pi}{5} $.}
    x_a(t + T) &= 3\cos\left(5t +\textcolor{red}{\cancelto{\textcolor{black}{2\pi}}{\textcolor{black}{5\frac{2\pi}{5}}}} + \frac{\pi}{6}\right) \\
    &= 3\cos\left(\textcolor{red}{\cancel{\textcolor{black}{2\pi}}} + \left(5t + \frac{\pi}{6}\right)\right) = 3cos\left(5t + \frac{\pi}{6}\right)\\
    \intertext{The addition of $ 2\pi $ is redundant since it represents a full rotation. Therefore, $ x_a(t + T) = x_a(t) $ and $ x_a(t) $ is periodic.}
\end{eqnsection}
\newpage
\begin{equation*}
    x(n) = 3cos\left(5n + \frac{\pi}{6}\right)
\end{equation*}
\vspace{1em}\\

\begin{eqnsection}{-3em}{b) Consider the expression to determine if $x(n)$ is periodic:}
    \intertext{We can expand $x(n + N)$ as follows:}
 x(n+N) &= 3\cos\left(5n + 5N + \frac{\pi}{6}\right) \\
 \intertext{For a discrete signal, the frequency is defined as $f_0 = \frac{k}{N}$:}
 \intertext{since $2\pi f_0n = 5n \ \ \ f_0 = \frac{5}{2\pi} \ \ \ T = \frac{2\pi}{5}$}
 \intertext{We can stop here since we know a discrete-time signal is only periodic if the frequency $f_0$ can be expressed as a ratio of two integers. Since there is a factor of $\pi$ in the expression, this is aperiodic.}
\end{eqnsection} 
\vspace{1em}\\
\begin{eqnsection}{-3em}{c) Consider the following equations to determine if $x(n)$ is periodic:}
    x(n) &= 2e^{j\left(\frac{n}{6} - \pi\right)} = 2e^{-j\pi}e^{\frac{n}{6}} \\
    \intertext{A phase shift not dependent on n does not change the periodicity:}
    &= 2e^{\frac{n}{6}} = 2\left(cos\left(\frac{n}{6}\right) + jsin\left(\frac{n}{6}\right)\right) \\
    \intertext{since $2\pi f_0n = \frac{n}{6} \ \ \ f_0 = \frac{1}{12\pi} \ \ \ T = 12\pi$}
    \intertext{Again, we can stop here since we know a discrete-time signal is only periodic if the frequency $f_0$ can be expressed as a ratio of two integers. Since there is a factor of $\pi$ in the expression, this is aperiodic.}
\end{eqnsection}
\vspace{1em}\\
\begin{eqnsection}{-3em}{d) Consider the following equations to determine if $x(n)$ is periodic:}
    x(n) &= cos\left(\frac{n}{8}\right)cos\left(\frac{n\pi}{8}\right)\\
    \intertext{We have two frequencies to evaluate, $f_{0_0} = \frac{n}{8}$ and $f_{0_1} = \frac{\pi n}{8}$:}
\intertext{since $2\pi f_{0_i}n = \left[\frac{n}{8} , \frac{n\pi}{8}\right] \ \ \ f_{0_i} = \left[\frac{1}{16\pi} , \frac{1}{16}\right]  \ \ \ T_i = [16\pi, 16]$}
\intertext{While we came up with one expression for $f_{0_i}$ that is rational, the other one is not. $\therefore$ the overall signal will be aperiodic.}
\end{eqnsection}
\vspace{1em}\\
\begin{eqnsection}{-3em}{e) Consider the following equations to determine if $x(n)$ is periodic:}
    x(n) = cos\left(\pi \frac{n}{2}\right) - sin\left(\pi \frac{n}{8}\right) + 3cos\left(\pi \frac{n}{4} + \frac{\pi}{3}\right)\\
        \intertext{We have three frequencies to evaluate, $f_{0_0} = \frac{\pi n}{2}$, $f_{0_1} = \frac{\pi n}{8}$, and $f_{0_2} = \frac{\pi n}{4}$:}
\intertext{since $2\pi f_{0_i}n = \left[\frac{\pi n}{2} , \frac{n\pi}{8} , \frac{n\pi}{4}\right] \ \ \ f_{0_i} = \left[\frac{1}{4} ,\frac{1}{16}, \frac{1}{8}\right]  \ \ \ T_i = [4, 16, 8]$}
\intertext{Since the frequencies $f_{0_i}$ are all rational. The signal is periodic with $f_0 = \frac{1}{16}$.}
\end{eqnsection}
\vspace{1em}\\
\noindent
\section{}
1.7 An analog signal contains frequencies up to 10 kHz.

\begin{enumerate}
    \item What range of sampling frequencies allows exact reconstruction of this signal
from its samples?
    \item Suppose that we sample this signal with a sampling frequency Fs = 8 kHz.
Examine what happens to the frequency F1 = 5 kHz.
    \item Repeat part (b) for a frequency F2 = 9 kHz.
\end{enumerate}
\hrule
\vspace{3em}
\begin{eqnsection}{-3em}{a) What range of sampling frequencies allows exact reconstruction of this signal from its samples?}
\intertext{By $F_s > 2F_{max}$ we can find the range of frequencies:}
-\frac{F_s}{2} \leq F_{max} \leq \frac{F_s}{2}\\
-\frac{F_s}{2} \leq 10kHz \leq \frac{F_s}{2} \\
-F_s \leq 20kHz \leq F_s \\
\intertext{$\therefore$ the range is up to |20kHz|}
\end{eqnsection}

\begin{eqnsection}{-3em}{b and c) Suppose that we sample this signal with a sampling frequency Fs = 8 kHz. Examine what happens to the frequency F1 = 5 kHz. Repeat part (b) for a frequency F2 = 9 kHz.}
    \intertext{Sampling the signals if they are outside the range we found, at this frequency, would give us a mapping from that frequency to one within our range.}
    F_s = \left[5\ 9\right]kHz
    \intertext{looking at what happens in the outputted frequencies will show:}
        F_i\ mod (F_N) &\rightarrow{} F_i - F_N - F_N \ \ \  \text{i.f.f.}\ \ F_i > F_N\\
     else\  F_i\ mod (F_N)&\rightarrow{} F_i \ \ \  \text{if}\ \ F_i < F_N\\ 
     \intertext{This will give the mapping of the frequencies and our $F_N = \frac{{F_{s}}}{2} = 4kHz$:}
     F_i\ mod (F_N) = \left[5\ 9\right]kHz\\
     \intertext{Both 5 and 9 kHz are above the maximum frequency we can capture, $\therefore F_i > F_N$}
      = \left[5\ 9\right]kHz - 2F_N \\
     = \left[-3\ \ -1\right]kHz \\
     \intertext{$\therefore$ both frequencies are mapped to others within the range.}
\end{eqnsection}


\section{}
1.9 - An analog signal $x_a(t) = sin(480\pi t) + 3 sin(720\pi t)$ is sampled 600 times per second.

\begin{enumerate}
    \item Determine the Nyquist sampling rate for $x_a(t)$.
    \item Determine the folding frequency.
    \item What are the frequencies, in radians, in the resulting discrete time signal $x(n)$?
    \item If $x(n)$ is passed through an ideal D/A converter, what is the reconstructed signal
$y_a(t)$?
\end{enumerate}
\vspace{3em}\hrule
\begin{eqnsection}{-3em}{a and b) Determine the Nyquist sampling rate for $x_a(t)$.}
    \intertext{$F_{max}$ of $360Hz$ gives a Nyquist rate of $2F_{max} = 720Hz$}
    \intertext{sampling at $600Hz$ gives a Folding Freq of $\frac{F_s}{2} = 300Hz$}
\end{eqnsection}
\begin{eqnsection}{-3em}{What are the frequencies, in radians, in the resulting discrete time signal $x(n)$?}
    F_i &= \left[240\ 360\right]\\
    F_1 &= 240\ \  F_1 < F_N \rightarrow{} F_1 = 240\\
    F_2 &= 360\ \  F_2 > F_N \rightarrow{} F_2 = 60 - 300 = -240\\ 
    \intertext{$\therefore$ 360 maps to -240}
    F_i &= \pm(2\pi240) = \pm(480\pi)\\
\end{eqnsection}
\newpage
\begin{eqnsection}{-3em}{If $x(n)$ is passed through an ideal D/A converter, what is the reconstructed signal $y_a(t)$?}
    x_a(t) &= sin(480\pi t) + 3 sin(720\pi t)\\
    y_a(t) &= sin(480\pi t) + 3 sin(-480\pi t)\\
    y_a(t) &= sin(480\pi t) - 3 sin(480\pi t)\\
    y_a(t) &= - 2 sin(480\pi t)    
\end{eqnsection}
\vspace{2em}
\hrule
\section{}
1.10 - A digital communication link carries binary-coded words representing samples of an
input signal:
\begin{equation*}
    x_a(t) = 3 cos(600\pi t) + 2 cos(1800\pi t)
\end{equation*}
The link is operated at 10,000 bits/s, and each input sample is quantized into 1024
different voltage levels.

\begin{enumerate}
  \item What are the sampling frequency and the folding frequency?
  \item What is the Nyquist rate for the signal $x_a(t)$?
  \item What are the frequencies in the resulting discrete-time signal $x(n)$?
  \item What is the resolution $\Delta$?
\end{enumerate}
\hrule
\vspace{3em}
\begin{eqnsection}{-3em}{What are the sampling frequency and the folding frequency?}
    \intertext{To find this, we need to use the 1024 levels as a clue. With bits of data, we use the set $B = \{0\ 1\}$, meaning we need to use the $log_2$ function to find the number of bits we need to represent the data:}
    log_2(1024) &= 10\\ 
    \intertext{With this number of bits, we can divide the operating rate by this to find the frequency of captures.}
    BPS &= rate*bits \\
    rate &= F_s = \frac{BPS}{bits} = \frac{1\textcolor{red}{\cancel{\textcolor{black}{0}}}k\textcolor{red}{\cancel{\textcolor{black}{B}}}}{s\textcolor{red}{\cancel{\textcolor{black}{10B}}}} = \frac{1k}{s} = 1kHz \\ 
    F_f &= \frac{F_s}{2} = 500Hz
\end{eqnsection}
\newpage
\begin{eqnsection}{-3em}{What is the Nyquist rate for the signal $x_a(t)?$}
    \intertext{The signal has a max $\omega_i$ of $1800\pi$:}
    1800\pi &= 2\pi900 \\ 
    \therefore \text{Nyquist rate} &= 2F_{max} = 1800Hz
\end{eqnsection}
\vspace{2em}\\
\begin{eqnsection}{-3em}{What are the frequencies in the resulting discrete-time signal $x(n)$?}
    \intertext{Frequencies in the signal would be:}
    f_i &= \left[300\ \ 900\right]Hz\\
    f_{i_n} &= \frac{f_i n}{f_f} = \left[\frac{300n}{500}\ \ \frac{900n}{500}\right]Hz\\
    &= \left[\frac{3n}{5}\ \ \frac{-1n}{5}\right]Hz\\
    \therefore \omega_i &= \left[-\frac{1\pi n}{5}\ \ \frac{3\pi n}{5}\right]\\
\end{eqnsection}
\begin{eqnsection}{-3em}{What is the resolution $\Delta$?}
    \intertext{$\Delta$ represents the quantization of the signal using the number of levels we have available to us. Additionally, we need to find the min and max of the signal we have.}
    f_i &= \left[300\ \ 900\right]Hz\\
   2\pi f_i &= \left[2\pi300\ \ 2\pi900\right] = \left[2\pi300\ \ 3(2\pi300)\right]
   \intertext{Note: the second frequency here is an integer multiple of the first. $\therefore$ the max will happen at the same time every other rotation. by adding a phaseshift of $\pi$ to both would make the min line up in the signal as well.}
   \therefore x_{max} &= 3 + 2 = 5\\
   x_{min} &= -3 + -2 = -5\\ 
   \Delta &= \frac{x_{max} - x_{min}}{L - 1} = \frac{5 - (-5)}{1024 - 1} \approx 9.8e^{-3}V \\ \therefore \Delta &= 9.8mV
\end{eqnsection}

\end{document}
